In an easy-plane ferroelectric chain with no external field, the polarization angle $\Theta(x,t)$ obeys the damped sine–Gordon equation $\Theta_{tt}+\alpha\Theta_t-c^2\Theta_{xx}+\sin\Theta=0$ for $x\in\mathbb{R}$, where $\alpha>0$ and $c>0$. Assume zero topological number so $\Theta(+\infty,t)=\Theta(-\infty,t)\in2\pi\mathbb{Z}$ for all $t$. The chain is prepared at rest with a uniform state: $\Theta_t(x,0)=0$, $\Theta(x,0)=\tfrac{7\pi}{8}$. Determine whether $\lim_{t\to+\infty}\Theta(x,t)=2\pi$ holds or whether the system relaxes to a different asymptotic state.

Reduce to the uniform ODE: spatially uniform data keep $\Theta(x,t)=\vartheta(t)$, so $\vartheta_{tt}+\alpha\vartheta_t+\sin\vartheta=0$ with $\vartheta(0)=\tfrac{7\pi}{8}$ and $\vartheta_t(0)=0$. The energy $E(t)=\tfrac12\vartheta_t^2+1-\cos\vartheta$ satisfies $\dot E=-\alpha\vartheta_t^2\le0$, hence $E$ decreases and the motion must settle at a stationary point with $\sin\vartheta=0$ where the stable equilibria are $2\pi k$. Here $\vartheta_{tt}(0)=-\sin(\tfrac{7\pi}{8})<0$, so $\vartheta$ initially decreases, and $E(0)=1-\cos(\tfrac{7\pi}{8})=1+\cos(\tfrac{\pi}{8})<2=1-\cos\pi$ shows the trajectory has insufficient energy to surmount the barrier at $\pi$. Therefore it cannot climb to $2\pi$ and instead relaxes to the nearest stable minimum below, so $\lim_{t\to+\infty}\Theta(x,t)=0$.