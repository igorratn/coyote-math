Fix two (half-)integer angular momenta $j_1,j_2\ge2$ and consider the Clebsch–Gordan coefficient $$f(j_3)=\langle j_1,0;j_2,0\mid j_3,0\rangle$$ for all allowed $j_3$ satisfying the triangle inequality $|j_1-j_2|\le j_3\le j_1+j_2$ and the parity condition that $j_1+j_2+j_3$ is an integer. Introduce the discrete variable $$x=j_1+j_2-j_3,$$ so that $x$ runs over the integers $$x=0,1,\dots,2\min(j_1,j_2).$$ Regard $f$ as a function of $x$: $$f(x)=\langle j_1,0;j_2,0\mid j_1+j_2-x,0\rangle.$$ Determine, with proof, whether there exists at least one integer $x$ with $$0<x<2\min(j_1,j_2)$$ such that $f(x)=0$.

For fixed $j_1,j_2$ and $m_1=m_2=m_3=0$, the family of coefficients $f(x)$ as $j_3$ (equivalently $x$) varies forms an eigenvector of the SU(2) Casimir operator acting in the coupled basis. In the $j_3$ (or $x$) representation this eigenproblem becomes a three–term recurrence relation in $x$ with polynomial coefficients, which is exactly the recurrence satisfied by dual Hahn polynomials (see Biedenharn & Louck, _Angular Momentum in Quantum Physics_, Cambridge Univ. Press, 1981, Ch. 9). Thus, up to an overall scale factor, $f(x)$ is a dual Hahn polynomial in the discrete variable $x=j_1+j_2-j_3$ of degree $N=\min(j_1,j_2)$, with $x$ taking integer values $0,1,\dots,2N$.

General theory of dual Hahn polynomials as an orthogonal family on a finite uniform grid implies that a degree–$N$ dual Hahn polynomial has exactly $N$ simple zeros, all located strictly inside the open interval $(0,2N)$; this follows from the standard zero-distribution results for orthogonal polynomials on finite discrete sets (cf. Szegő, _Orthogonal Polynomials_, AMS, 1975, Ch. III). Since $f(x)$ is proportional to such a degree–$N$ dual Hahn polynomial, it must vanish at exactly $N=\min(j_1,j_2)$ distinct integers $x$ satisfying

$$0<x<2\min(j_1,j_2).$$
In particular, there exists at least one integer $x$ with $0<x<2\min(j_1,j_2)$ such that $f(x)=0$, as required.