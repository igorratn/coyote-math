In the attached $(u,v)$-chart of a surface, the marked point $P$ lies on the coordinate circle $u=\text{const}$, oriented by increasing $v$. At $P$ it is given that $\sinh u=\tfrac{4}{3}$. Compute the geodesic curvature $k_{g}$ at $P$ of this circle, oriented by increasing $v$. Give your final answer as an exact number (no decimals).


Step 1. Work on the surface with metric $ds^{2}=du^{2}+\sinh^{2}u\,dv^{2}$. The curve is $u=\text{const}$ oriented by increasing $v$.

Step 2. For orthogonal coordinates with $F=0$ the geodesic curvature of a $v$–curve is $k_g=\tfrac{1}{\sqrt{EG}}\,\partial_u(\sqrt{G})=\tfrac{1}{\sinh u}\,\partial_u(\sinh u)=\tfrac{\cosh u}{\sinh u}$.

Step 3. At $P$ we have $\sinh u=\tfrac{4}{3}$. Using $\cosh^{2}u-\sinh^{2}u=1$ gives $\cosh u=\sqrt{1+\left(\tfrac{4}{3}\right)^{2}}=\tfrac{5}{3}$. Hence $k_g=\tfrac{5}{4}$.

Final answer $5/4$


In the attached $(u, v)$-chart of a surface embedded as the upper sheet of the hyperboloid $-x^2 - y^2 + z^2 = 1$ with parameterization $x = \sinh u \cos v$, $y = \sinh u \sin v$, $z = \cosh u$, the marked point $P$ lies on a curve indicated in the figure, oriented by increasing $v$. At $P$, it is given that $\sinh u = \frac{4}{3}$. Compute at $P$ the geodesic curvature $k_g$, the normal curvature $k_n$, and the geodesic torsion $\tau_g$ of this curve, in that order. Give your final answer as a simplified fractional triplet.
