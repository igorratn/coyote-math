The Racah polynomials $\hat{R}_n(\lambda(x))$ are the most general family of discrete orthogonal polynomials, defined on a quadratic lattice $\lambda(x)$, and are orthogonal on a finite set of $N+1$ points.

We consider the sequence of orthonormal polynomials $\{\hat{R}_n(\lambda(x))\}_{n=0}^N$, where $\|\hat{R}_n\|^2 = 1$ for all $n$. These polynomials satisfy the three-term recurrence relation (TTRR):

$$\lambda(x) \hat{R}_n(\lambda(x)) = \alpha_n \hat{R}_{n+1}(\lambda(x)) + \beta_n \hat{R}_n(\lambda(x)) + \gamma_n \hat{R}_{n-1}(\lambda(x))$$

The TTRR holds for the full range $n=0, 1, \dots, N$ by imposing the boundary conditions:

-   $\hat{R}_{-1}(\lambda(x)) \equiv 0$ (This requires $\gamma_0 = 0$ for $n=0$).
    
-   $\hat{R}_{N+1}(\lambda(x)) \equiv 0$ (This requires $\alpha_N = 0$ for $n=N$).
    

The TTRR is represented by the finite $(N+1) \times (N+1)$ Jacobi matrix $\mathbf{J}_{N+1}$.

Rigorously determine whether the following statement is true or false:
The Jacobi matrix $\mathbf{J}_{N+1}$ is necessarily Hermitian (symmetric in the real case) because its eigenvalues are the real, distinct points of the quadratic lattice $\{\lambda(x)\}$; and furthermore, this symmetry requires the recurrence coefficients to satisfy the condition: $\mathbf{\alpha_n = \gamma_{n+1}}$ for all $n=0, 1, \dots, N-1$.

The statement is FALSE.

The justification provided for the matrix symmetry is invalid.

The argument claims: (Matrix has real, distinct eigenvalues) $\implies$ (Matrix must be symmetric). This is a flawed causal link in linear algebra.

Real eigenvalues are necessary, but not sufficient, to prove a matrix is symmetric. The matrix $\mathbf{J}_{N+1}$ is actually symmetric because it represents the self-adjoint multiplication operator $M_{\lambda}$ in an orthonormal basis $\{\hat{R}_n\}$.

Since the premise used is logically unsound, the entire statement is FALSE. (The final conclusion $\alpha_n = \gamma_{n+1}$ is mathematically correct, but for the reason stated above, not the one given).