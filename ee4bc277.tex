In a one-dimensional mechanical chain of coupled pendula without external drive, the angular displacement field $\Theta(x,t)$ satisfies the damped sine–Gordon equation $\Theta_{tt}+\alpha\Theta_t-c^2\Theta_{xx}+\sin\Theta=0$ for $x\in\mathbb{R}$, where $\alpha>0$ represents weak damping and $c>0$ is the wave speed. Assume zero topological (winding) number, $\Theta(+\infty,t)=\Theta(-\infty,t)\in2\pi\mathbb{Z}$ for all $t$. The chain is prepared at rest with a uniform initial deflection $\Theta_t(x,0)=0$ and $\Theta(x,0)=\tfrac{7\pi}{4}$. Determine whether $\lim_{t\to+\infty}\Theta(x,t)=0$ holds, or whether the pendulum chain relaxes to a different asymptotic state.

Reduce to the uniform ODE: spatially uniform data keep $\Theta(x,t)=\vartheta(t)$, so $\vartheta_{tt}+\alpha\vartheta_t+\sin\vartheta=0$ with $\vartheta(0)=\tfrac{7\pi}{4}$ and $\vartheta_t(0)=0$. The energy $E(t)=\tfrac12\vartheta_t^2+1-\cos\vartheta$ satisfies $\dot E=-\alpha\vartheta_t^2\le0$, so $E$ decreases and the motion must settle at a stationary point with $\sin\vartheta=0$; stable equilibria are $2\pi k$. Here $\vartheta_{tt}(0)=-\sin(\tfrac{7\pi}{4})=+\tfrac{\sqrt2}{2}>0$, so $\vartheta$ initially increases toward $2\pi$, and $E(0)=1-\cos(\tfrac{7\pi}{4})=1-\tfrac{\sqrt2}{2}<2$ shows it cannot cross any barrier to lower wells. Thus the chain relaxes to the nearest stable minimum above, not to $0$, and $\lim_{t\to+\infty}\Theta(x,t)=2\pi$.