Let **$\{{p_n(x)}\}_{n=0}^\infty$** be orthonormal polynomials with respect to a positive weight **$w(x)$** on **$[a,b]$**, satisfying the recurrence

$$x p_n(x)=a_{n+1}p_{n+1}(x)+b_n p_n(x)+a_n p_{n-1}(x)\quad(a_{n+1}>0, b_n\in\mathbb R).$$

For fixed **$x_0\in(a,b)$** and **$N\ge1$**, consider the polynomial

$$\pi_t(x)=p_{N+1}(x)-t p_N(x).$$

Determine (with proof) whether the following two quantities are equal:

$$\inf_{t\in\mathbb R}\frac{\|\pi_t\|_w^2}{\pi_t(x_0)^2} \quad\text{and}\quad \inf_{\substack{Q\in\mathbb P_N\\Q(x_0)\ne0}}\frac{\|Q\|_w^2}{Q(x_0)^2}.$$

The two quantities are not equal in general.

Proof


For a fixed integer $N\ge1$ and parameter $t\in\mathbb R$, we first define the polynomial

$$\pi_t(x)=p_{N+1}(x)-t p_N(x),\qquad x\in(a,b).$$

Then, for this fixed $N$ and a fixed point $x_0\in(a,b)$, we compare the two infima.

Let $A=p_N(x_0)$, $B=p_{N+1}(x_0)$. By orthonormality,
$$
\|\pi_t\|_w^2=1+t^2,\qquad \pi_t(x_0)=B-tA.
$$
Consider
$$
F(t)=\frac{1+t^2}{(B-tA)^2}.
$$

By the interlacing theorem for zeros of consecutive orthogonal polynomials (G. Szegő, Theorem 3.3.2, _Orthogonal Polynomials_, Fourth Edition, AMS Colloquium Publications, Vol. 23, 1975), the polynomials $p_N(x)$ and $p_{N+1}(x)$ cannot both vanish at an interior point $x_0 \in (a,b)$, therefore $A^2+B^2>0$.


- If $A=0$ (then $B\neq0$), $F(t)=\frac{1+t^2}{B^2}$ is minimised at $t=0$ with value $1/B^2$.

- If $A\neq0$, compute
  $$
  F'(t)=\frac{2t(B-tA)^2+2A(1+t^2)(B-tA)}{(B-tA)^4}.
  $$
  The numerator is $2(B-tA)(tB+A)$. Setting it to zero and discarding the root $B-tA=0$ (where $F(t)=+\infty$) yields the unique critical point $t_*=-A/B$. Direct substitution gives
  $$
  F(t_*)=\frac{1}{A^2+B^2}.
  $$
  As $|t|\to\infty$, $F(t)\sim1/A^2>1/(A^2+B^2)$, so $t_*$ is the global minimum.

Thus in all cases
$$
\inf_t\frac{\|\pi_t\|_w^2}{\pi_t(x_0)^2}=\frac{1}{p_N(x_0)^2+p_{N+1}(x_0)^2}.
$$

The second quantity is the minimum weighted norm squared for any polynomial $Q$ of degree at most $N$ normalized to $Q(x_0)=1$.

$$\inf_{\substack{Q\in\mathbb P_N\\Q(x_0)\ne0}}\frac{\|Q\|_w^2}{Q(x_0)^2}$$

This quantity is defined as the Christoffel function, $\lambda_N(x_0)$, and its value is the reciprocal of the Christoffel-Darboux kernel evaluated at $x_0$:

$$\lambda_N(x_0)=\frac{1}{K_N(x_0,x_0)} = \frac{1}{\sum_{j=0}^N p_j(x_0)^2}$$

The two quantities, $\inf_{t} \frac{\|\pi_t\|_w^2}{\pi_t(x_0)^2}$ and $\inf_{Q} \frac{\|Q\|_w^2}{Q(x_0)^2}$, are equal if and only if their denominators are equal:

Expanding the sum, this requires:

$$p_N(x_0)^2+p_{N+1}(x_0)^2 = \left(\sum_{j=0}^{N-1} p_j(x_0)^2\right) + p_N(x_0)^2$$

Which simplifies to the condition:

$$p_{N+1}(x_0)^2 = \sum_{j=0}^{N-1} p_j(x_0)^2$$

This relation is not true in general.

Counterexample (orthonormal Legendre polynomials). The problem assumes the polynomials $\{p_n(x)\}$ are orthonormal with respect to a positive weight $w(x)$ on an interval $[a,b]$ and satisfy a three-term recurrence. On the interval $[-1,1]$ with weight $w(x)=1$, the orthonormal Legendre polynomials are

$$p_n(x)=\sqrt{\frac{2n+1}{2}}\,P_n(x),$$

where $P_n$ are the standard Legendre polynomials. They satisfy the classical three-term recurrence

$$(n+1)P_{n+1}(x)=(2n+1)xP_n(x)-nP_{n-1}(x),$$

which, after orthonormalization, is of the form

$$x p_n(x)=a_{n+1}p_{n+1}(x)+b_n p_n(x)+a_n p_{n-1}(x),\quad a_{n+1}>0,\ b_n\in\mathbb R,$$

so all assumptions of the prompt are satisfied.

Choose $N=1$ and $x_0=0$. Using $P_0(0)=1$, $P_1(0)=0$, $P_2(0)=-\tfrac12$ gives

$$p_0(0)^2=\tfrac12,\quad p_1(0)^2=0,\quad p_2(0)^2=\tfrac58.$$

From the general formulas,

$$\inf_{t\in\mathbb R}\frac{\|\pi_t\|_w^2}{\pi_t(0)^2} =\frac{1}{p_1(0)^2+p_2(0)^2} =\frac{1}{0+5/8} =\frac85,$$

and, since the second infimum is over $\mathbb P_1$,

$$\inf_{\substack{Q\in\mathbb P_1\\Q(0)\ne0}}\frac{\|Q\|_w^2}{Q(0)^2} =\frac{1}{p_0(0)^2+p_1(0)^2} =\frac{1}{1/2+0} =2.$$

Thus in this Legendre example the two quantities are $8/5$ and $2$, so they are not equal in general.

This completes the proof.\centering