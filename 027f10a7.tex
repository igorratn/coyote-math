Let $P_n^{(\alpha,\beta)}(x)$ denote the Jacobi polynomials, orthogonal on $[-1,1]$ with respect to $p(x)=(1-x)^\alpha(1+x)^\beta$, where $\alpha,\beta>-1$. 

Define the modified weight $\tilde p(x)=(1-x)^{\alpha+1}(1+x)^\beta=(1-x)p(x)$, and suppose that for $n\ge1$, $\tilde P_n^{(\alpha,\beta)}(x)=P_n^{(\alpha,\beta)}(x)+\delta_nP_{n-1}^{(\alpha+1,\beta)}(x)$, with $\deg(\tilde P_n^{(\alpha,\beta)})=n$. 
The polynomials $\tilde P_n^{(\alpha,\beta)}(x)$ are constructed to be orthogonal with respect to $\tilde p(x)$.

Claim. The condition $(\tilde P_n^{(\alpha,\beta)},P_{n-1}^{(\alpha+1,\beta)})_{\tilde p}=0$ is necessary and sufficient to determine $\delta_n$. 

Determine whether this claim is **True or False**, and give a rigorous proof of your conclusion.

**The claim is True.**

Necessity

The weight $\tilde{p}(x) = (1-x)^{\alpha+1}(1+x)^\beta$ (with $\alpha > -1$, $\beta > -1$) is nonnegative on $[-1,1]$ and positive on $(-1,1)$. Any degree-$n$ polynomial orthogonal to all lower-degree polynomials under $\langle \cdot, \cdot \rangle_{\tilde{p}}$ must vanish on inner products with any degree $< n$ polynomial, including the degree $n-1$ $P_{n-1}^{(\alpha+1,\beta)}(x)$. Hence,

$$\langle \tilde{P}_n^{(\alpha,\beta)}, P_{n-1}^{(\alpha+1,\beta)} \rangle_{\tilde{p}} = 0.$$

Sufficiency

Imposing the condition yields

$$\delta_n = -\frac{\langle P_n^{(\alpha,\beta)}, P_{n-1}^{(\alpha+1,\beta)} \rangle_{\tilde{p}}}{\|P_{n-1}^{(\alpha+1,\beta)}\|_{\tilde{p}}^2},$$

unique since the norm is positive.

The family $\{P_k^{(\alpha+1,\beta)}(x)\}_{k \geq 0}$ is orthogonal under $\tilde{p}$ (standard Jacobi with parameters $\alpha+1, \beta$) and spans polynomials of degree $< n$ for $k = 0$ to $n-1$.

For $k \leq n-2$,

$$\langle \tilde{P}_n^{(\alpha,\beta)}, P_k^{(\alpha+1,\beta)} \rangle_{\tilde{p}} = \langle P_n^{(\alpha,\beta)}, P_k^{(\alpha+1,\beta)} \rangle_{\tilde{p}} + \delta_n \langle P_{n-1}^{(\alpha+1,\beta)}, P_k^{(\alpha+1,\beta)} \rangle_{\tilde{p}}.$$

1.  The second term is zero by orthogonality under $\tilde{p}$.
    
2.  The first is $\langle P_n^{(\alpha,\beta)}, (1-x) P_k^{(\alpha+1,\beta)} \rangle_p$ (original weight $p(x) = (1-x)^\alpha (1+x)^\beta$), and $\deg((1-x) P_k^{(\alpha+1,\beta)}) \leq n-1$, so it vanishes by orthogonality of $P_n^{(\alpha,\beta)}$ under $p$.
    

The case $k = n-1$ holds by the condition.

Thus, full orthogonality to degrees $< n$ is achieved.

**Final Answer: True.**