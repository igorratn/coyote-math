Let $P_n^{(\alpha,\beta)}(x)$ denote the Jacobi polynomials, orthogonal on $[-1,1]$ with respect to $p(x)=(1-x)^\alpha(1+x)^\beta$, where $\alpha,\beta>-1$.

Fix $\beta_1\in\mathbb R$ with $|\beta_1|>1$ and define the modified weight

$$\tilde p(x)=\frac{p(x)}{x-\beta_1}.$$

For functions $f$ and $g$, define the inner product

$$(f,g)_{\tilde p}=\int_{-1}^1 f(x)g(x)\tilde p(x)dx.$$

For $n\ge1$, define a sequence of polynomials $\tilde P_n^{(\alpha,\beta)}(x)$ by the ansatz

$$\tilde P_n^{(\alpha,\beta)}(x)=P_n^{(\alpha,\beta)}(x)+\gamma_n P_{n-1}^{(\alpha,\beta)}(x),$$

so that $\deg(\tilde P_n^{(\alpha,\beta)})=n$ for every $\gamma_n\in\mathbb R.$

Claim: The condition $(\tilde P_n^{(\alpha,\beta)},1)_{\tilde p}=0$ is necessary and sufficient for

$$(\tilde P_n^{(\alpha,\beta)},q)_{\tilde p}=0 \quad \text{for every polynomial } q \text{ with } \deg(q)<n.$$


Determine whether this claim is True or False, and give a rigorous proof of your conclusion.

Conclusion: True.

Proof.

Necessity. If $(\tilde P_n^{(\alpha,\beta)},q)_{\tilde p}=0$ for every polynomial $q$ with $\deg(q)<n$, then taking $q=1$ gives $(\tilde P_n^{(\alpha,\beta)},1)_{\tilde p}=0$.

Sufficiency. Assume $(\tilde P_n^{(\alpha,\beta)},1)_{\tilde p}=0$. Let $q$ be any polynomial with $\deg(q)<n$ and write $q(x)=q(\beta_1)+(x-\beta_1)r(x)$, where $r(x)=(q(x)-q(\beta_1))/(x-\beta_1)$ is a polynomial with $\deg(r) \le n-2$. Then

$$(\tilde P_n,q)_{\tilde p}=\int_{-1}^1 \tilde P_n(x)q(x)\frac{p(x)}{x-\beta_1}dx=q(\beta_1)(\tilde P_n,1)_{\tilde p}+\int_{-1}^1 \tilde P_n(x)r(x)p(x)dx.$$

The first term is zero by assumption. For the second term, using $\tilde P_n=P_n+\gamma_n P_{n-1}$ and $\deg(r) \le n-2$, Jacobi orthogonality with respect to $p$ gives $(P_n,r)_p=0$ and $(P_{n-1},r)_p=0$, hence $\int_{-1}^1 \tilde P_n(x)r(x)p(x)dx=0$. Therefore $(\tilde P_n,q)_{\tilde p}=0$ for all $q$ with $\deg(q)<n$.

This proves $(\tilde P_n,1)_{\tilde p}=0$ is necessary and sufficient for full orthogonality to all polynomials of degree less than $n$ with respect to $\tilde p$.

Final Answer: True.aLet $P_n^{(\alpha,\beta)}(x)$ denote the Jacobi polynomials, orthogonal on $[-1,1]$ with respect to $p(x)=(1-x)^\alpha(1+x)^\beta$, where $\alpha,\beta>-1$.

Fix $\beta_1\in\mathbb R$ with $|\beta_1|>1$ and define the modified weight

$$\tilde p(x)=\frac{p(x)}{x-\beta_1}.$$

For functions $f$ and $g$, define the inner product

$$(f,g)_{\tilde p}=\int_{-1}^1 f(x)g(x)\tilde p(x)dx.$$

For $n\ge1$, define a sequence of polynomials $\tilde P_n^{(\alpha,\beta)}(x)$ by the ansatz

$$\tilde P_n^{(\alpha,\beta)}(x)=P_n^{(\alpha,\beta)}(x)+\gamma_n P_{n-1}^{(\alpha,\beta)}(x),$$

so that $\deg(\tilde P_n^{(\alpha,\beta)})=n$ for every $\gamma_n\in\mathbb R.$

Claim: The condition $(\tilde P_n^{(\alpha,\beta)},1)_{\tilde p}=0$ is necessary and sufficient for

$$(\tilde P_n^{(\alpha,\beta)},q)_{\tilde p}=0 \quad \text{for every polynomial } q \text{ with } \deg(q)<n.$$


Determine whether this claim is True or False, and give a rigorous proof of your conclusion.

Conclusion: True.

Proof.

Necessity. If $(\tilde P_n^{(\alpha,\beta)},q)_{\tilde p}=0$ for every polynomial $q$ with $\deg(q)<n$, then taking $q=1$ gives $(\tilde P_n^{(\alpha,\beta)},1)_{\tilde p}=0$.

Sufficiency. Assume $(\tilde P_n^{(\alpha,\beta)},1)_{\tilde p}=0$. Let $q$ be any polynomial with $\deg(q)<n$ and write $q(x)=q(\beta_1)+(x-\beta_1)r(x)$, where $r(x)=(q(x)-q(\beta_1))/(x-\beta_1)$ is a polynomial with $\deg(r) \le n-2$. Then

$$(\tilde P_n,q)_{\tilde p}=\int_{-1}^1 \tilde P_n(x)q(x)\frac{p(x)}{x-\beta_1}dx=q(\beta_1)(\tilde P_n,1)_{\tilde p}+\int_{-1}^1 \tilde P_n(x)r(x)p(x)dx.$$

The first term is zero by assumption. For the second term, using $\tilde P_n=P_n+\gamma_n P_{n-1}$ and $\deg(r) \le n-2$, Jacobi orthogonality with respect to $p$ gives $(P_n,r)_p=0$ and $(P_{n-1},r)_p=0$, hence $\int_{-1}^1 \tilde P_n(x)r(x)p(x)dx=0$. Therefore $(\tilde P_n,q)_{\tilde p}=0$ for all $q$ with $\deg(q)<n$.

This proves $(\tilde P_n,1)_{\tilde p}=0$ is necessary and sufficient for full orthogonality to all polynomials of degree less than $n$ with respect to $\tilde p$.

Final Answer: True.