Let $\mathbb{Q}$ be the set of all rational numbers, and let $A={x\in\mathbb{Q}:x^2<2}$. Prove or disprove that $A$ has a least upper bound.

Proof.
Let $\mathbb{Q}$ be the set of all rational numbers and $A={x\in\mathbb{Q}:x^2<2}$.
Every $x\in A$ satisfies $x<\sqrt{2}$, so $A$ is bounded above in $\mathbb{R}$, for example by $\sqrt{2}$ itself or by any number greater than $\sqrt{2}$.
The number $\sqrt{2}$ is an upper bound of $A$ because if $x^2<2$, then $x<\sqrt{2}$.
If $\alpha<\sqrt{2}$, then $\alpha^2<2$, and since the rational numbers are dense in $\mathbb{R}$, there exists a rational $x$ such that $\alpha<x<\sqrt{2}$.
This $x$ belongs to $A$ and is greater than $\alpha$, so $\alpha$ is not an upper bound of $A$.
Hence $\sqrt{2}$ is the smallest real number that bounds $A$ above.
Therefore, $\sup A=\sqrt{2}$.