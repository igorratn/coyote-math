In the attached image, a flat torus is represented by a rhombic fundamental domain with $60^\circ$ basis vectors. Point $P$ is shown, and several candidate paths labeled $A,B,C,D,E$ are drawn from $P$ toward different lattice lifts of a point $Q$. The Voronoi diagram of the lifts of $Q$ is also included. Based on this diagram, which labeled path represents the true geodesic on the torus? Give your final answer as one of $A,B,C,D,E$.

Step 1
On a flat torus, the minimizing geodesic from $P$ to $Q$ is the straight segment in the plane from $P$ to the nearest lattice lift of $Q$ (e.g., $Q^\prime, Q^{\prime\prime},\dots$). Its length is the Euclidean distance $d=\sqrt{(\Delta x)^2+(\Delta y)^2}$ in the unfolded tiling.

Step 2
In the figure, $Q$, $Q^\prime$, and $Q^{\prime\prime}$ have (visibly) the same vertical level as each other, so $\Delta y$ from $P$ to each of these lifts is the same. Therefore the shortest segment is the one with the smallest horizontal offset $|\Delta x|$.

Step 3
By inspection, the horizontal offset from $P$ to $Q^\prime$ is smaller than the offset from $P$ to $Q$ (and much smaller than to $Q^{\prime\prime}$). Hence $|PQ^\prime|<|PQ|<|PQ^{\prime\prime}|$, and $A$ (the segment to $Q^\prime$) is the unique minimizing geodesic among the labeled options.

Final Answer → A