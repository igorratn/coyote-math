In a long Josephson junction (or any continuum limit of a coupled-pendulum chain), the gauge-invariant phase $\Theta(x,t)$ obeys the damped sine–Gordon equation
$\Theta_{tt}+\alpha\Theta_t-\Theta_{xx}+\sin\Theta=0$ on $x\in\mathbb{R}$,
where $\alpha>0$ represents weak dissipation. Assume no external drive and that the topological (winding) number is initially zero. The system is prepared at rest with a uniform near-hilltop bias $\Theta_t(x,0)=0$ and $\Theta(x,0)=3\pi-\varepsilon$, where $\varepsilon>0$ is arbitrarily small and spatially uniform. Determine whether $\lim_{t\to+\infty}\Theta(x,t)=2\pi$ holds, or whether the field relaxes to a different asymptotic state.

For the damped sine–Gordon system $\Theta_{tt}+\alpha\Theta_t-\Theta_{xx}+\sin\Theta=0$ with $\alpha>0$, $\Theta(x,0)=3\pi-\varepsilon$, and $\Theta_t(x,0)=0$, spatially uniform initial data imply $\Theta(x,t)=\vartheta(t)$ satisfying the single-pendulum equation $\vartheta_{tt}+\alpha\vartheta_t+\sin\vartheta=0.$ Since $\vartheta(0)=3\pi-\varepsilon$ lies slightly below an unstable equilibrium ($3\pi$), the restoring torque $\sin\vartheta<0$ drives $\vartheta$ downward. Dissipation removes kinetic energy, so $\vartheta(t)$ cannot cross the next barrier at $4\pi$ and must approach the nearest stable minimum at $2\pi.$ Hence, $$\lim_{t\to+\infty}\Theta(x,t)=2\pi.$$ The result follows from the monotone decay of the total energy $E(t)=\tfrac12\vartheta_t^2+1-\cos\vartheta$, with $\dot E=-\alpha\vartheta_t^2\le0$, which forces the system to settle in the potential minimum closest to the initial condition.