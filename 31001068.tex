
A physical quantity $Q_n$ is directly proportional to the value of a high-order Racah polynomial $u_n^{(\alpha,\beta)}(x)$ at the discrete argument $x_0$. 
You are given $a=100$, $b=10000$, and $x_0=5025$. 
Determine whether the following statement is true or false: the corresponding continuous argument is $t_0=-\frac{99}{200}$.

The statement is TRUE.

The calculation relies on the algebraic transformation that connects the discrete argument $x$ of the Racah polynomial to the continuous argument $t$ of the approximating Jacobi polynomial, a necessary step when the defining parameters are large ($b=10,000$).

The asymptotic mapping formula between the discrete Racah argument $x$ and the continuous Jacobi argument $t$ is established as:

$$x = \frac{b}{2} - \frac{a}{2} t + \frac{1}{4}$$

(Reference: Nikiforov, A.F., Suslov, S.K., & Uvarov, V.B., _Classical Orthogonal Polynomials of a Discrete Variable_, 1991, Chapter 3, Section 3.8, page 111)

We rearrange the formula to solve for the continuous argument $t_0$:

$$\frac{a}{2} t_0 = \frac{b}{2} + \frac{1}{4} - x_0$$

$$t_0 = \frac{2}{a} \left( \frac{b}{2} + \frac{1}{4} - x_0 \right)$$

We substitute the given values: $b=10,000$, $a=100$, and $x_0=5,025$:

$$t_0 = \frac{2}{100} \left( \frac{10,000}{2} + \frac{1}{4} - 5,025 \right)$$

$$t_0 = \frac{1}{50} \left( 5,000 + 0.25 - 5,025 \right)$$

$$t_0 = \frac{1}{50} \left( -24.75 \right)$$

Converting the result to a simplified fraction:

$$-24.75 = - \frac{99}{4}$$

$$t_0 = \frac{1}{50} \left( -\frac{99}{4} \right)$$

$$t_0 = -\frac{99}{200}$$

Since the calculated value $t_0 = -\frac{99}{200}$ matches the value stated in the question, the statement is TRUE.