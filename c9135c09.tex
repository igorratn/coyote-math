In an anisotropic ferromagnetic spin chain, the azimuthal spin angle $\Theta(x,t)$ satisfies the damped sine–Gordon equation $\Theta_{tt}+\alpha\Theta_t-\Theta_{xx}+\sin\Theta=0$ for $x\in\mathbb{R}$, where $\alpha>0$ represents Gilbert damping. Assume no external magnetic drive and zero topological (winding) number, so $\Theta(\pm\infty,t)\in2\pi\mathbb{Z}$ for all $t$. The chain is prepared at rest with a uniform magnetization: $\Theta_t(x,0)=0$, $\Theta(x,0)=\tfrac{3\pi}{4}$. Determine whether $\lim_{t\to+\infty}\Theta(x,t)=2\pi$ holds, or whether the spin field relaxes to a different asymptotic state.

Because the data are spatially uniform, $\Theta(x,t)=\vartheta(t)$ and $\vartheta$ solves $\vartheta_{tt}+\alpha\vartheta_t+\sin\vartheta=0$ with $\vartheta(0)=\tfrac{3\pi}{4}$, $\vartheta_t(0)=0$. The energy $E(t)=\tfrac12\vartheta_t^2+1-\cos\vartheta$ satisfies $\dot E(t)=-\alpha\vartheta_t^2\le0$, so $E$ decreases and the motion approaches an equilibrium. Stable equilibria are $2k\pi$; the barrier at $\pi$ has height $V(\pi)-V(0)=2$. Here $E(0)=1-\cos(\tfrac{3\pi}{4})=1+\tfrac{\sqrt2}{2}<2$, so the trajectory cannot reach $\pi$ and thus cannot go to the $2\pi$ well. Moreover $\vartheta_{tt}(0)=-\sin(\tfrac{3\pi}{4})=-\tfrac{\sqrt2}{2}<0$, so it initially moves toward $0$. Therefore $\lim_{t\to+\infty}\Theta(x,t)=0\neq2\pi$.