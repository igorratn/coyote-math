In a torsional DNA model (Yakushevich-type) without external torque, the base-pair twist angle $\Theta(x,t)$ satisfies the damped sine–Gordon equation $\Theta_{tt}+\gamma\Theta_t-c^2\Theta_{xx}+\sin\Theta=0$ for $x\in\mathbb{R}$, where $\gamma>0$ models viscous drag from the solvent and $c>0$ is the torsional wave speed. Assume zero topological (winding) number, so $\Theta(+\infty,t)=\Theta(-\infty,t)\in2\pi\mathbb{Z}$ for all $t$. The chain is prepared at rest with a uniform overtwist: $\Theta_t(x,0)=0$, $\Theta(x,0)=3\pi+\varepsilon$, where $\varepsilon>0$ is small and spatially uniform. Determine whether $\lim_{t\to+\infty}\Theta(x,t)=2\pi$ holds, or whether the twist relaxes to a different asymptotic state.

Reduce to the uniform ODE: spatially uniform data keep $\Theta(x,t)=\vartheta(t)$, so $\vartheta_{tt}+\gamma\vartheta_t+\sin\vartheta=0$ with $\vartheta(0)=3\pi+\varepsilon$, $\vartheta_t(0)=0$, $\varepsilon>0$. The energy $E(t)=\tfrac12\vartheta_t^2+1-\cos\vartheta$ satisfies $\dot E=-\gamma\vartheta_t^2\le0$ so $E$ is strictly decreasing and the motion must approach an equilibrium with $\sin\vartheta=0$ i.e. $\vartheta\in\pi\mathbb Z$. At $t=0$, $\vartheta_{tt}(0)=-\sin(3\pi+\varepsilon)=+\sin\varepsilon>0$ so the angle initially increases, rolling off the unstable peak at $3\pi$ toward the next minimum $4\pi$. Moreover $E(0)=1-\cos(3\pi+\varepsilon)=1+\cos\varepsilon<2=1-\cos(3\pi)$, so the trajectory does not have enough energy to surmount the next barrier at $5\pi$ and damping prevents any later escape from the $4\pi$ well. With zero winding number the uniform limit at $4\pi$ is admissible. Therefore $\lim_{t\to+\infty}\Theta(x,t)=4\pi$, not $2\pi$.