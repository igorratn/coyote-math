\documentclass[tikz,border=12pt]{standalone} % Increased border for better visibility
\usepackage{amsmath}
\usetikzlibrary{calc}

\begin{document}
\begin{tikzpicture}[scale=5]

% --- 60° lattice basis: a=(1,0), b=(0.5, sqrt(3)/2) ---
\def\bx{1}
\def\by{0}
\def\cx{0.5}
\def\cy{0.8660254}

% Points P and (base) Q in the central cell
\coordinate (P) at (0.20,0.25);
\coordinate (Q) at (0.75,0.70);

% Lattice lifts Q_{(m,n)} = Q + m*a + n*b
\coordinate (Qm10)   at ($(Q)+(-1,0)$);          % (-1,0)
\coordinate (Q10)    at ($(Q)+(1,0)$);          % ( 1,0)
\coordinate (Q01)    at ($(Q)+(\cx,\cy)$);       % ( 0,1)
\coordinate (Q0m1)  at ($(Q)-(\cx,\cy)$);       % ( 0,-1)
\coordinate (Qm1m1) at ($(Q)+(-1,0)-(\cx,\cy)$);  % (-1,-1)

% Removed the conflicting 3x3 rectangular tiling.

% Central fundamental domain (Rhombus)
\draw[line width=.6pt] (0,0)--(1,0)--(1.5,0.8660254)--(0.5,0.8660254)--cycle;

% Draw Voronoi cells (Hexagons centered at the LIFT points Q_(i,j))
% This corrects the geometric alignment.
\coordinate (HexV1) at (0.5, 0);
\coordinate (HexV2) at (0.25, 0.433);
\coordinate (HexV3) at (-0.25, 0.433);
\coordinate (HexV4) at (-0.5, 0);
\coordinate (HexV5) at (-0.25, -0.433);
\coordinate (HexV6) at (0.25, -0.433);
\foreach \i in {-1,0,1}{
  \foreach \j in {-1,0,1}{
    % Center Voronoi cell on the lift Q + L_(i,j)
    \draw[line width=0.5pt, black, dashed]
      ($(Q)+(\i*\bx+\j*\cx, \i*\by+\j*\cy)+(HexV1)$) --
      ($(Q)+(\i*\bx+\j*\cx, \i*\by+\j*\cy)+(HexV2)$) --
      ($(Q)+(\i*\bx+\j*\cx, \i*\by+\j*\cy)+(HexV3)$) --
      ($(Q)+(\i*\bx+\j*\cx, \i*\by+\j*\cy)+(HexV4)$) --
      ($(Q)+(\i*\bx+\j*\cx, \i*\by+\j*\cy)+(HexV5)$) --
      ($(Q)+(\i*\bx+\j*\cx, \i*\by+\j*\cy)+(HexV6)$) -- cycle;
  }
}

% Points (do NOT draw base Q to avoid implying comparisons)
\fill (P) circle (.015) node[below left=2pt] {$P$};
\fill (Qm10)  circle (.012) node[left=2pt]     {$Q_{(-1,0)}$};
\fill (Q10)   circle (.012) node[right=2pt]      {$Q_{(1,0)}$};
\fill (Q01)   circle (.012) node[above=2pt]      {$Q_{(0,1)}$};
\fill (Q0m1)  circle (.012) node[below=2pt]      {$Q_{(0,-1)}$};
\fill (Qm1m1) circle (.012) node[below left=2pt]  {$Q_{(-1,-1)}$};

% Candidate paths with distinct colors but same style
\draw[line width=1pt, dashed, blue]   (P)--(Qm10)  node[midway, left=2pt]   {A};
\draw[line width=1pt, dashed, red]    (P)--(Q01)   node[midway, above=1pt]  {B};
\draw[line width=1pt, dashed, green!70!black] (P)--(Q10)   node[midway, below=2pt, right=2pt]  {C};
\draw[line width=1pt, dashed, purple] (P)--(Q0m1)  node[midway, below=1pt, left=2pt]  {D};
\draw[line width=1pt, dashed, orange] (P)--(Qm1m1) node[midway, left=2pt]   {E};

\end{tikzpicture}
\end{document}