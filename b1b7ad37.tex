In a quasi-one-dimensional charge-density-wave conductor with no external bias, the CDW phase $\Theta(x,t)$ satisfies the damped sine–Gordon equation $\Theta_{tt}+\Gamma\Theta_t-v^2\Theta_{xx}+\sin\Theta=0$ for $x\in\mathbb{R}$, where $\Gamma>0$ models damping and $v>0$ is the phase velocity. Assume zero topological (winding) number, so $\Theta(+\infty,t)=\Theta(-\infty,t)\in2\pi\mathbb{Z}$ for all $t$. The system is prepared at rest with a uniform phase: $\Theta_t(x,0)=0$, $\Theta(x,0)=5\pi+\tfrac{\pi}{4}$. Determine whether $\lim_{t\to+\infty}\Theta(x,t)=4\pi$ holds, or whether the phase relaxes to a different asymptotic state.

Reduce to the uniform ODE: with spatially uniform data and no drive, $\Theta(x,t)=\vartheta(t)$ solves $\vartheta_{tt}+\Gamma\vartheta_t+\sin\vartheta=0$ with $\vartheta(0)=5\pi+\tfrac{\pi}{4}$ and $\vartheta_t(0)=0$. The mechanical energy $E(t)=\tfrac12\vartheta_t^2+V(\vartheta)$ with $V(\vartheta)=1-\cos\vartheta$ satisfies $\dot E=-\Gamma\vartheta_t^2\le0$, so $E$ is nonincreasing. Initial acceleration: $\vartheta_{tt}(0)=-\sin(5\pi+\tfrac{\pi}{4})=+\sin(\tfrac{\pi}{4})>0$, hence $\vartheta$ initially increases toward $6\pi$. Energy position: $E(0)=V(5\pi+\tfrac{\pi}{4})=1-\cos(\tfrac{\pi}{4}+\pi)=1-(-\tfrac{\sqrt2}{2})=1+\tfrac{\sqrt2}{2}<2=V(5\pi)$, so no barrier must be surmounted moving from $5\pi$ toward the well at $6\pi$. With damping, trajectories approach equilibria $\sin\vartheta=0$; the stable minimum reached in the direction of motion is $6\pi$. Therefore $\lim_{t\to+\infty}\Theta(x,t)=6\pi$.