In the attached image of a flat torus shown as a tiled unit square with opposite edges identified, several candidate paths from $P$ to $Q$ are drawn and labeled. Identify which path represents the true geodesic on the torus, and give your answer by referring to the correct label directly from the figure.

Step 1. The flat torus is formed by identifying opposite edges of a unit square, creating a tiled plane where $P$ and $Q$ are marked in the central square, with a nearby image $Q'$ in an adjacent tile. Paths are labeled: $A$ (straight to $Q'$), $B$ (curved), $C$ (broken), and $D$ (straight right wrap).

Step 2. Geodesics on a flat torus are straight lines in the tiled plane; curved or broken paths are not.

Step 3. $A$ is the only straight path to the closest $Q'$, making it the shortest route. $B$, $C$, and $D$ are longer due to curvature, bends, or excessive wrapping.
Final Answer. The geodesic is path $A$.