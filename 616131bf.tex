n the attached image, determine the geodesic curvature, the normal curvature, and the geodesic torsion of the intersection curve at the marked point 
P
P. Give your final answer as a simplified fractional triplet in the order (geodesic curvature, normal curvature, geodesic torsion).


**Step 1 (curve & point).**

  

From the image the surface is $z=xy$ and the plane is $x=1$. Their intersection is $\gamma(t)=(1,t,t)$. The marked point is $P=(1,1,1)$, which corresponds to $t=1$.

  

  

**Step 2 (tangent).**

  

We have $\gamma^{\prime}(t)=(0,1,1)$. At $t=1$, $\lVert \gamma^{\prime}(1)\rVert=\sqrt{2}$, so the unit tangent is $T=\bigl(0,\tfrac{1}{\sqrt{2}},\tfrac{1}{\sqrt{2}}\bigr)$. Since $\gamma^{\prime\prime}(t)=0$, the space curvature vanishes, so both geodesic curvature and normal curvature are $0$.

  

  

**Step 3 (surface normal).**

  

With parametrization $r(u,v)=(u,v,uv)$, the unit normal is $N(u,v)=\dfrac{(-v,-u,1)}{\sqrt{u^{2}+v^{2}+1}}$. Along the curve we have $N(t)=\dfrac{(-t,-1,1)}{\sqrt{t^{2}+2}}$. At $t=1$, this gives $N=\dfrac{(-1,-1,1)}{\sqrt{3}}$.

  

  

**Step 4 (surface binormal).**

  

The surface binormal is $U=N\times T=\bigl(-\tfrac{2}{\sqrt{6}},\tfrac{1}{\sqrt{6}},-\tfrac{1}{\sqrt{6}}\bigr)$.

  

  

**Step 5 (normal’s $s$-derivative).**

  

Differentiate $N(t)$. At $t=1$, we obtain $N^{\prime}(1)=\dfrac{(-2,1,-1)}{3\sqrt{3}}$. Since $\tfrac{ds}{dt}=\lVert\gamma^{\prime}(t)\rVert=\sqrt{2}$, we get $N_{s}=\dfrac{N^{\prime}(1)}{\sqrt{2}}=\dfrac{(-2,1,-1)}{3\sqrt{6}}$.

  

  

**Step 6 (geodesic torsion).**

  

From Darboux we have $N_{s}=-\kappa_{n}T-\tau_{g}U$, so $\tau_{g}=-\langle N_{s},U\rangle$. Compute

  

$\langle N_{s},U\rangle=\dfrac{(-2)(-2)+(1)(1)+(-1)(-1)}{3\cdot 6}=\dfrac{6}{18}=\tfrac{1}{3}$.

  

Thus $\tau_{g}=-\tfrac{1}{3}$.

**Final Answer:**  $(0,0,-\tfrac{1}{3})$.