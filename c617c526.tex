Consider the second-order linear difference equation on the non-uniform lattice $x(k)=k(k+\gamma+\delta+1)$, $k\in{0,1,\dots,N}$, $\sigma(k)\Delta^2P_n(k)+\tau(k)\Delta P_n(k)+\lambda_nP_n(k)=0,$ where $\Delta$ is the forward difference operator, and suppose that when this equation is rewritten in terms of the lattice variable $x=x(k)$, the polynomial solutions $P_n(x)$ form the Racah family. Question: Decide, with rigorous justification, whether the following statement is true or false: for Racah polynomial solutions on the lattice $x(k)=k(k+\gamma+\delta+1)$, it is necessary and sufficient that the coefficients, viewed as functions of $x$, satisfy $\deg\sigma(x)\le2$ and $\deg\tau(x)\le1$.

The statement is TRUE.

The necessary and sufficient condition for a family of polynomials $P_n(x)$ of exact degree $n$ to satisfy a second–order linear difference equation

$$\sigma(x)\Delta\nabla P_n(x)+\tau(x)\Delta P_n(x)+\lambda_n P_n(x)=0$$

with coefficients independent of $n$ is that

$$\deg\sigma(x)\le2,\qquad \deg\tau(x)\le1.$$

This classification theorem is given in Nikiforov, Suslov & Uvarov, Classical Orthogonal Polynomials of a Discrete Variable, Springer, 1991.

(See the presentation of hypergeometric-type difference equations and the classification of classical discrete orthogonal polynomials in the introductory theory.)

For the quadratic lattice

$$x(k)=k(k+\gamma+\delta+1),$$

this condition is both necessary and sufficient for the polynomial solutions to be the Racah polynomials or one of their limiting cases.

Therefore the required maximal degrees are

$$\deg\sigma=2,\qquad \deg\tau=1,$$

and the statement is TRUE.