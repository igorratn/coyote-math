In the attached image, the unit disk is equipped with the hyperbolic metric shown. The geodesic triangle $\triangle OPQ$ is drawn with sides along geodesics: $OP$ and $OQ$ are diameters of the disk, and $PQ$ is a circular arc orthogonal to the boundary. By construction, the arc $PQ$ is chosen so that the triangle is symmetric and the base angles are equal. Determine the hyperbolic area of $\triangle OPQ$. Give your final answer as a rational multiple of $\pi$ in lowest terms.

**Step 1.** We are in the Poincaré disk (conformal model), so Euclidean and hyperbolic angles coincide at interior points. The diameters meet orthogonally at $O$, hence $\angle O = \tfrac{\pi}{2}$.

 **Step 2.** The picture is symmetric across $y=x$, so the base angles are equal: $\angle P = \angle Q = \theta$.

  **Step 3.** Let $P=(a,0)$ and $Q=(0,a)$. For the geodesic arc $PQ$ (circle orthogonal to the boundary) with center $C$ on $y=x$, the angle at $P$ satisfies

$\cos \theta = \dfrac{a^2+1}{2\sqrt{a^4+1}}$.

In this construction the symmetry and orthogonality fix $a^2 = 2 - \sqrt{3}$, so $\cos \theta = \tfrac{\sqrt{3}}{2}$ and thus $\theta = \tfrac{\pi}{6}$.

  **Step 4.** The triangle’s angles are $\angle O = \tfrac{\pi}{2}$, $\angle P = \tfrac{\pi}{6}$, $\angle Q = \tfrac{\pi}{6}$. By Gauss–Bonnet in curvature $-1$, the area is

$\pi - \big(\tfrac{\pi}{2} + \tfrac{\pi}{6} + \tfrac{\pi}{6}\big) = \tfrac{\pi}{6}$.